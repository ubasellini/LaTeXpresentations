%%%%%%%%%%%%%%%%%%%%%%%%%%%%%%%%%%%%%%%%%%%%%%%%%%%%%%%%%%%%%%
%%															%%
%% 															%%
%%     PRESENTATIONS IN LATEX %% PRESENTATIONS IN LATEX     %% 
%%     PRESENTATIONS IN LATEX %% PRESENTATIONS IN LATEX  	%%
%%												   			%%
%% 		Template file for the 2018/2019 EDSD cohort 		%%
%%															%%
%% 		Start date:   29/01/2018							%%
%%		Last update:  29/01/2019							%%
%% 		Author: Ugofilippo Basellini						%%
%% 		Email:  ugofilippo.basellini@ined.fr				%%
%%															%%
%%															%%
%%%%%%%%%%%%%%%%%%%%%%%%%%%%%%%%%%%%%%%%%%%%%%%%%%%%%%%%%%%%%%

%%%%% declare your documentclass as beamer
\documentclass[11pt]{beamer}

%%%%% choose your presentation's background colors.
%%%%% For a combination of themes and color themes, see:
%%%%% https://hartwork.org/beamer-theme-matrix/
\mode<presentation>{
	\usetheme{Warsaw}     	 % choose your presentation theme 
	\usecolortheme{beaver}  % choose your color theme
	% \setbeamercovered{transparent}   % comment out if you want visible overlays	
}

%% load some standard packages 
\usepackage[english]{babel}
\usepackage[latin1]{inputenc}
\usepackage{mathptmx}
\usepackage{textpos}
\usepackage{setspace}      % Modify line spacing

%% your presentation title (short and long)
\title[Short Title]{My first presentation in \LaTeX}

%% your presentation subtitle (if you have one)
\subtitle{Learning the \texttt{BEAMER} class} 

%% authors
\author[Author1, Author2] % (optional, use only with lots of authors)
{F.~Author\inst{1} \and S.~Author\inst{2}}
% - Give the names in the same order as the appear in the paper.
% - Use the \inst{?} command only if the authors have different
%   affiliation.

%% institutions
\institute[Universities XX and YY] % (optional, but mostly needed)
{
	\inst{1}%
	Department of F.~Author \\
	University XX
	\and
	\inst{2}%
	Department of S.~Author \\
	University YY}
% - Use the \inst command only if there are several affiliations.
% - Keep it simple, no one is interested in your street address.

%% date
\date[EDSD 2018-2019] % (optional, should be abbreviation of conference name)
{Conference - Date}
% - Either use conference name or its abbreviation.
% - Not really informative to the audience, more for people (including
%   yourself) who are reading the slides online

% Table of contents options (uncomment if you need to use the TOC):
%\AtBeginSection[]{
%	\begin{frame}<beamer>{Outline}
%	\tableofcontents[currentsection,currentsubsection]
%	\end{frame}
%}

%% add a LOGO in the upper right corner
\addtobeamertemplate{frametitle}{}{
	\begin{textblock*}{100mm}(.91\textwidth,-1.79cm)
		\includegraphics[scale=0.17]{EDSDLogo}
	\end{textblock*}
}

%% do not show the navigation bar
\beamertemplatenavigationsymbolsempty

%% add frame numbers
\setbeamertemplate{footline}
{
	\leavevmode%
	\hbox{%
		\begin{beamercolorbox}[wd=.5\paperwidth,ht=2.25ex,dp=1ex,center]{author in head/foot}%
			\usebeamerfont{author in head/foot}\insertshortauthor
		\end{beamercolorbox}%
		\begin{beamercolorbox}[wd=.5\paperwidth,ht=2.25ex,dp=1ex,center]{title in head/foot}%
			\usebeamerfont{title in head/foot}\insertshorttitle~~~~~~~~~~~~~~~~~~~~~~~~~~~~~~~~~~~~~~~~~~~~~~~~~~~~~~~~~\insertframenumber
		\end{beamercolorbox}%		
	}%
}


%% START YOUR DOCUMENT
\begin{document}

%% TITLE FRAME 
\begin{frame}    % First frame
\titlepage
\end{frame}	

%% TABLE OF COMMENTS FRAME
%% uncomment if you want to see the table of comments
%\begin{frame}{Outline}
%\tableofcontents
%% You might wish to add the option [pausesections]
%\end{frame}
	

%%%%%%%%%%%%%%%%%%%%%%%%%%%%%%%%%%%%%%%%%%%%%%%%%%%%%%%%%%%%%%%%%%%%%%%%%
%%%%%%%%%%%%%%%%%%%%%%%%%%%%%%%%%%%%%%%%%%%%%%%%%%%%%%%%%%%%%%%%%%%%%%%%%	
\section{Motivation}   % First Section

%%%%%%%%%%%%%%%%%%%%%%%%%%%%%%%%%%%%%%%%%%%%%%%%%%%%%%%%%%%%%%%%%%%%%%%%%
\begin{frame}          % Second frame
\frametitle{Introduction}
\begin{itemize}
\setlength\itemsep{2em}		
\item Background facts:
	\begin{itemize}
	\item it is well known that ...
	\item xx is a relevant issue ...
	\end{itemize}
\item Short overview of literature
	\begin{itemize}
	\item yy have shown that ...
	\item zz demonstrated that ...
	\end{itemize}
\end{itemize}
\end{frame}

%%%%%%%%%%%%%%%%%%%%%%%%%%%%%%%%%%%%%%%%%%%%%%%%%%%%%%%%%%%%%%%%%%%%%%%%%
\begin{frame}          % Third frame
\frametitle{The Research Questions}
Our main research questions are:
\bigskip
\begin{enumerate}
\setlength\itemsep{2em}	
\item is there any relationship between xx and yy?
\item if so, has the relationship changed over time? 
\end{enumerate}
\end{frame}

%%%%%%%%%%%%%%%%%%%%%%%%%%%%%%%%%%%%%%%%%%%%%%%%%%%%%%%%%%%%%%%%%%%%%%%%%
%%%%%%%%%%%%%%%%%%%%%%%%%%%%%%%%%%%%%%%%%%%%%%%%%%%%%%%%%%%%%%%%%%%%%%%%%	
\section{Methodology}   % Second Section

%%%%%%%%%%%%%%%%%%%%%%%%%%%%%%%%%%%%%%%%%%%%%%%%%%%%%%%%%%%%%%%%%%%%%%%%%	
\subsection{The data}   % Optional

%%%%%%%%%%%%%%%%%%%%%%%%%%%%%%%%%%%%%%%%%%%%%%%%%%%%%%%%%%%%%%%%%%%%%%%%%
\begin{frame}          % Fourth frame
\frametitle{The XYZ dataset}
%\framesubtitle{Subtitles are optional}
Here you can talk about your data. \\
\bigskip
Suggestions:
\begin{itemize}
\setlength\itemsep{2em}	
\item Use \texttt{itemize} or \texttt{enumerate} a lot.
\item Use very short sentences or short phrases.
\end{itemize}
\end{frame}

%%%%%%%%%%%%%%%%%%%%%%%%%%%%%%%%%%%%%%%%%%%%%%%%%%%%%%%%%%%%%%%%%%%%%%%%%	
\subsection{The model}    % Optional

%%%%%%%%%%%%%%%%%%%%%%%%%%%%%%%%%%%%%%%%%%%%%%%%%%%%%%%%%%%%%%%%%%%%%%%%%
\begin{frame}          % Fifth frame
\frametitle{The model}
Here you can talk about your model, including formulas and figures
\bigskip
\begin{equation}          
y = {\color{blue}\beta_0} + \beta_1x + \epsilon\,,\,\,
\epsilon \sim \mathcal{N}(0,\,\sigma^{2})
\end{equation}
\end{frame}

%%%%%%%%%%%%%%%%%%%%%%%%%%%%%%%%%%%%%%%%%%%%%%%%%%%%%%%%%%%%%%%%%%%%%%%%%
%%%%%%%%%%%%%%%%%%%%%%%%%%%%%%%%%%%%%%%%%%%%%%%%%%%%%%%%%%%%%%%%%%%%%%%%%	
\section{Results}   % Third Section

%%%%%%%%%%%%%%%%%%%%%%%%%%%%%%%%%%%%%%%%%%%%%%%%%%%%%%%%%%%%%%%%%%%%%%%%%
\begin{frame}          % Sixth frame
\frametitle{The Results}
You may want to use a block to highlight your results 
\bigskip
\begin{block}{Here is your block title}
	\begin{itemize}
	\setlength\itemsep{1.5em}	
	\item<1-> We have observed that ...
	\item<2-> It is also interesting to notice that ...  
	\end{itemize}
\end{block}
\end{frame}

%%%%%%%%%%%%%%%%%%%%%%%%%%%%%%%%%%%%%%%%%%%%%%%%%%%%%%%%%%%%%%%%%%%%%%%%%
\begin{frame}          % Seventh frame
\frametitle{The Results}
And you can include an image: 
\bigskip
\begin{center}
\includegraphics[scale=.7]{EDSDLogo}
\end{center}
\end{frame}

%%%%%%%%%%%%%%%%%%%%%%%%%%%%%%%%%%%%%%%%%%%%%%%%%%%%%%%%%%%%%%%%%%%%%%%%%
%%%%%%%%%%%%%%%%%%%%%%%%%%%%%%%%%%%%%%%%%%%%%%%%%%%%%%%%%%%%%%%%%%%%%%%%%	
\section{Conclusions}   % Fourth Section

%%%%%%%%%%%%%%%%%%%%%%%%%%%%%%%%%%%%%%%%%%%%%%%%%%%%%%%%%%%%%%%%%%%%%%%%%
\begin{frame}          % Eigth frame
\frametitle{Summary}
You can create overlays \dots
\begin{itemize}
\setlength\itemsep{1.2em}	
\item using the \texttt{pause} command:
	\begin{itemize}
	\setlength\itemsep{0.5em}	
	\item First item.
	\pause
	\item Second item.
	\end{itemize}
\item using overlay specifications:
	\begin{itemize}
	\setlength\itemsep{0.5em}	
	\item<3-> First item.
	\item<4-> Second item.
	\end{itemize}
\item<5-> using the general \texttt{uncover} command:
	\begin{itemize}
	\setlength\itemsep{0.5em}	
	\uncover<5->{\item First item.}
	\uncover<6->{\item Second item.}
	\end{itemize}
\end{itemize}

\end{frame}

%%%%%%%%%%%%%%%%%%%%%%%%%%%%%%%%%%%%%%%%%%%%%%%%%%%%%%%%%%%%%%%%%%%%%%%%%
\begin{frame}          % Last frame
\frametitle{$\,$}
\vspace{0.1cm}
\begin{center}
	\begin{LARGE}
		Do not forget to thank the audience!!
	\end{LARGE}
\end{center}
\bigskip
\end{frame}

\end{document}